\chapter{Seri Hat Hata Ayıklama Portu (SWD-DP)}

%Serial Wire Debug (SWD) is a 2-pin (SWDIO/SWCLK) electrical alternative JTAG interface that has the same JTAG protocol on top.
%SWD uses an ARM CPU standard bi-directional wire protocol, defined in the ARM Debug Interface v5.
%This enables the debugger to become another AMBA bus master for access to system memory and peripheral or debug registers.
%
%
%
%The Debug Access Port (DAP) is split into two main control units. The Debug Port (DP) and the Access Port (AP),
%and the physical connection to the debugger is part of the DP. The DAP supports two types of access,
%Debug Port (DP) accesses and Access Port (AP) accesses. External device to communicate directly with Serial Wire Debug Port
%(SW-DP) over SWDIO/SCLK pins. And SW-DP in turn can access one or several Access Ports (APs) the give access to the rest of
%the system. The MEM-AP is important AP which provide a way to access all memory
%and peripheral registers residing on the internal AHB/APB buses.

\acrfull{SWD}, iki pin bağlantısı ile, JTAG arayüzüne alternatif olarak, pin kısıtlaması olan mikrodenetleyiciler için geliştirilmiştir.
ARM Hata Ayıklama Arayüzü v5 ile tanımlanmış olan SWD, hata ayıklma portu ile ARM işlmecisinin \acrshort{AMBA} veriyoluna erişim sağlar. Bu sayade
işlmecinin sistem hafıza birimlerine, çevre birimlerinin ve sistem yazmaçlarına erişim sağlamaktadır.

\acrfull{DAP}, Hata Ayıklma Portu (\acrshort{DP}) ve Erişim Portu (\acrshort{AP}) olmak üzere iki ana kontrol birimine arılmıştır. ARM işlecisine erişmek isteyen cihaz
Seri Hat Hata Ayıklama Portu'nun SWDIO ve SCLK pinlerine fiziksel olarak bağlanaması gerekmektedir. SWD-DP ile sistemin geri kalan erişim ve hata ayıklama portuna erişim
sağlamak mümkündür. \acrfull{MEM-AP}, dahili AHB/APB veriyollarına, hafıza elemanlarına ve çevre birimlerine erişim sağlayabilen en önemli erişim portudur.

%---------------------------------- resim koy------------------------------------------------------------------

\section{SWD Protokolü}

SWD protokülü iki pin üzerinden (SWDIO-SCLK) hedef işlemcinin hata ayıklama portuna erişim sağlamaktadır. SWD'nin kullandığı iki pinden birisi olan SWDIO pini iki yönlü hat olup
veri alış-verişinin yapıldığı veri hattır. Diğer pin olan (SCLK), iletişim için gerekli olan saat (clock) sinyalini barındıran sinyal hattıdır. Saat sinyali hattı servis sunucusu tarafından
sağlanır.

\section{SWD Baglatı ve Reset}

Hedef cihazın Hata Ayıklama Portu'na fiziksel olarak bağlanıldıktan sonra, SWDIO pini lojik 1 seviyesine çekilerek SCLK hattından en az 50 puls verilmelidir. Bu işleme hat sıfırlama (Line Reset) adı verilir.
ARM işlemcisinide bulunan SWD Hata Ayıklama portunu seçmek için hat sıfırlama işleminden sonra onaltı bitlik "0xE79E" verisi, veri hattından saat sinyali ile birlikte gönderilir. Ardından
işlemcinin 'IDCODE' yazmacı okunur. Eğer okunan yazmaç değeri doğru ise SWD protoklü başarılı bir şekilde başlamış demektir.

\section{Veri Göndeme ve Alma Süreçleri}

SWD protokolü kullanarak . Bunlar;
\begin{itemize}
	\item İstek Fazı: Servis Sunucusu 8 bitlik istek paketini hedef cihaza gönderir.
	\item Doğrulama Fazı: Hedef cihaz 3 bitlik doğrulama kodunu servis sunucusuna gönderir.
	\item Veri Fazı: İstek fazında bulunan okuma-yazma isteğine bağlı olarak hedef cihaz yada sunucu cihaz 33 bitlik veriyi veri hattından gönderir.
\end{itemize}

%---------------------------- RESIM KOY pulsler ve data olan----------------------------------------------------------------

\subsection{İstek Fazı}

\subsubsection{DEneme}



