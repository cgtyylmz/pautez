\chapter{Mikrodenetleyicilerin Programlanması}

\acrfull{mcu}, içerisinde hafıza birimleri(RAM, ROM, Flash), giriş/çıkış birimleri olan programlanabilen entegrelerdir.
%%-_--_-_-_-_-_-_-_-_-_-_-_-_-_-_-_-Denetleyici Resimi Koy-_-_-_-_-_-_-_-_-_-_-_-_-_-_-_-_-_-_-_-_-_-_-_

Mikrodenetleyicilerin programlanması, bilgisayar oramında yazılmış olan kodun hedef işlemci için derlenmesi
ve derleme sonucu ortaya çıkan makina kodlarının bir şekilde denetleyicinin hafıza birimine aktarılması olarak açıklanır.
Mikrodenetlecilerden önce mikroişlemciler ile kurulan sistemlerde programlama işlemi, hafıza biriminin sistenden ayırarak
bir programlacı yardımı ile makina kodlarının aktarılmasıyla gerçekleşiyordu. Mikrodenetleyicilerde hafıza birimi entegre
içinde gömülü olarak bulunduğu için programalama işlemini bu şekilde gerçeklemek münkün değildir.

Günümüzlde mikrodenetleyiciler \acrfull{ICSP} olarak adlandırlan programlayıcılar ile programlanmaktadır. \arcshort{ICSP} programlayıcılar,
denetleyicinin programlama pinleri aracılıyla işlemcinin hafıza birimine erişim sağlayarak makina kodlarını aktarır.
Bu işlem ARM tabalı mikrodenetleyicilerde \acrfull{SWD} debug portu ya da JTAG portu ile gerçekleştirmektedir.

