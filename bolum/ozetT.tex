1900’lü yıllarda serüvenine başlayan güç elektroniği ve uygulamarı
yarıiletkenler ve sayısal elektronik ile büyük hız kazanmıştır. Güç elektroniği
uygulamarında gerekli olana tetikleme sinyali, giriş akımı, gerilimi; çıkış
akımı, gerilimi değerlerin ölçümesi gibi gereksinimleri mikrokontorolcüler
karşılamaktadır.

Günümüzün popüler ve hızla gelişen konularından olan Nesnelerin İnterneti
(IoT) ile cihazlar bir internete erişim sağlayıp diğer cihazlar ile iletişim
içeresinde bulunmaktadır. Bu gelişmeler, güç elektroniği uygulamarında
kullanılan mikrodenetleyicilerin uzaktan programlanması ve denetleyicilerin
topladığı verilerin internete aktarılması gibi yeni gereksinimleri ardında
getirmiştir.
