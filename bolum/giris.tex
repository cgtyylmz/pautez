\chapter{GİRİŞ}

Güç elektroniği serüvenine 1900'lü yıllarda güç elektroniği doğru akım motorlarının hız kontrolü ile başlamıştır.
Elektron tüpleri ile teorik çalışmalar yapılmıştır, fakat uygulamaya sokulmamıştır.
1950 yılında yarıiletkenler, 1960 yılında tristörler, 1980 yılında sayısal elektronik ve
mikroişlemcilerin geliştirilmesi ile güç elektroniği otomotiv, endüstri, ulaşım araçları gibi bir çok sektörde faailiyet
göstermektedir.

Güç elektroniğinde mikrodenetleyiciler, yarıiletkelerin ihtiyaç duyduğu tetikleme sinyallerinin oluştrulması,
akım ve gerilim değerlerin okunması gibi ihtiyaçları karşılmaktadır.

İletişim altyapısını ve teknik gelişmeler sonucu günümüzün popüler konusu olan nesnelerin interneti ile,
güç elektroniği uygulamarında kullananılan mikrodenetleyicilerin uzaktan erişim ile yazılımlarının ve paremetrelerinin
güncellenmesi, denetleyici tarafıdan okunan değerlerin görüntülenmesi gibi ihtiyaçlar ortaya çıkmıştır.

Bu çalışmada belirtilen ihtiyaçları karşılamak amacıyla; güç elektroniğinde sıklıkla kullanılan ARM tabanlı mikrodenetleyiciler
için programlamama ve veri aktarımı sağlamak üzere ESP8266 mikroişlemcisi kullanıldı.
ESP8266 hedef denetleyicinin debug portu olan \acrfull{SWD} portuna bağlı olacak ve buradan işemcinin
registerlarına ve hafıza birimlerine erişecek. Aynı zamanda web server olarak çalışacak ve kullanıcı ile hedef
işlemci arasında arayüz oluşturacaktır.



